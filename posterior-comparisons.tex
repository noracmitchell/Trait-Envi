\documentclass[12pt]{article}\usepackage[]{graphicx}\usepackage[]{color}
%% maxwidth is the original width if it is less than linewidth
%% otherwise use linewidth (to make sure the graphics do not exceed the margin)
\makeatletter
\def\maxwidth{ %
  \ifdim\Gin@nat@width>\linewidth
    \linewidth
  \else
    \Gin@nat@width
  \fi
}
\makeatother

\definecolor{fgcolor}{rgb}{0.345, 0.345, 0.345}
\newcommand{\hlnum}[1]{\textcolor[rgb]{0.686,0.059,0.569}{#1}}%
\newcommand{\hlstr}[1]{\textcolor[rgb]{0.192,0.494,0.8}{#1}}%
\newcommand{\hlcom}[1]{\textcolor[rgb]{0.678,0.584,0.686}{\textit{#1}}}%
\newcommand{\hlopt}[1]{\textcolor[rgb]{0,0,0}{#1}}%
\newcommand{\hlstd}[1]{\textcolor[rgb]{0.345,0.345,0.345}{#1}}%
\newcommand{\hlkwa}[1]{\textcolor[rgb]{0.161,0.373,0.58}{\textbf{#1}}}%
\newcommand{\hlkwb}[1]{\textcolor[rgb]{0.69,0.353,0.396}{#1}}%
\newcommand{\hlkwc}[1]{\textcolor[rgb]{0.333,0.667,0.333}{#1}}%
\newcommand{\hlkwd}[1]{\textcolor[rgb]{0.737,0.353,0.396}{\textbf{#1}}}%

\usepackage{framed}
\makeatletter
\newenvironment{kframe}{%
 \def\at@end@of@kframe{}%
 \ifinner\ifhmode%
  \def\at@end@of@kframe{\end{minipage}}%
  \begin{minipage}{\columnwidth}%
 \fi\fi%
 \def\FrameCommand##1{\hskip\@totalleftmargin \hskip-\fboxsep
 \colorbox{shadecolor}{##1}\hskip-\fboxsep
     % There is no \\@totalrightmargin, so:
     \hskip-\linewidth \hskip-\@totalleftmargin \hskip\columnwidth}%
 \MakeFramed {\advance\hsize-\width
   \@totalleftmargin\z@ \linewidth\hsize
   \@setminipage}}%
 {\par\unskip\endMakeFramed%
 \at@end@of@kframe}
\makeatother

\definecolor{shadecolor}{rgb}{.97, .97, .97}
\definecolor{messagecolor}{rgb}{0, 0, 0}
\definecolor{warningcolor}{rgb}{1, 0, 1}
\definecolor{errorcolor}{rgb}{1, 0, 0}
\newenvironment{knitrout}{}{} % an empty environment to be redefined in TeX

\usepackage{alltt}

%% for inline R code: if the inline code is not correctly parsed,
%% you will see a message
\newcommand{\rinline}[1]{SOMETHING WRONG WITH knitr}



\IfFileExists{upquote.sty}{\usepackage{upquote}}{}
\begin{document}

\title{Posterior comparison of regression coefficients}

\maketitle

\noindent The idea here is pretty simple: The credible intervals
associated with each coefficient in the regressions identifies
associations for which we have good evidence, i.e., those for which
the credible interval doesn't overlap zero. But we can't compare
the regression coefficients for {\it Protea\/} and {\it Pelargonium\/}
simply by comparing which coefficients are detectably different from
zero, except in the simple case when coefficients don't overlap zero and
have opposite signes.

It is, however, easy to construct statistics to compare them using
posterior comparisons~\cite{Holsinger-Wallace-2004}. We already have a
posterior sample for each set of regression coefficients, so we simply
select one coefficient from the posterior in the {\it Protea\/} model,
one from the posterior in the {\it Pelargonium\/} model, take the
difference, repeat a large number of times, calculate the credible
interval of the difference, and determine whether it overlaps zero. If
it does, we don't have evidence that the coefficients are
different. If it doesn't, we do.

%% read in the posterior results
\begin{knitrout}
\definecolor{shadecolor}{rgb}{0.969, 0.969, 0.969}\color{fgcolor}\begin{kframe}
\begin{alltt}
\hlkwd{load}\hlstd{(}\hlstr{"results-2014-03-06-09-01-02.Rsave"}\hlstd{)}
\hlstd{protea} \hlkwb{<-} \hlstd{fit}\hlopt{$}\hlstd{BUGSoutput}\hlopt{$}\hlstd{sims.list}
\hlkwd{load}\hlstd{(}\hlstr{"results-2014-03-08-13-25-09.Rsave"}\hlstd{)}
\hlstd{pellie} \hlkwb{<-} \hlstd{fit}\hlopt{$}\hlstd{BUGSoutput}\hlopt{$}\hlstd{sims.list}
\hlkwd{rm}\hlstd{(fit)}
\end{alltt}
\end{kframe}
\end{knitrout}

{\tt protea} and {\tt pellie} are now simple objects that contain the
posterior sample for each coefficient in the model. We're only
interested in comparing the {\tt beta.*} coefficients. Since the two
chains were run separately from one another and are of the same size
we can construct our posterior sample by simply running through each
sample fromt the posterior sequentially. First, we set up the
posterior comparison function. That's really easy (so long as we're
using symmetric credible intervals) because all we have to do is
determine the lower and upper bounds, take the difference of the input
vectors and return the quantiles.

\begin{knitrout}
\definecolor{shadecolor}{rgb}{0.969, 0.969, 0.969}\color{fgcolor}\begin{kframe}
\begin{alltt}
\hlstd{compare} \hlkwb{<-} \hlkwa{function}\hlstd{(}\hlkwc{x}\hlstd{,} \hlkwc{y}\hlstd{,} \hlkwc{prob} \hlstd{=} \hlnum{0.95}\hlstd{) \{}
  \hlstd{lo} \hlkwb{<-} \hlstd{(}\hlnum{1} \hlopt{-} \hlstd{prob)}\hlopt{/}\hlnum{2}
  \hlstd{hi} \hlkwb{<-} \hlnum{1} \hlopt{-} \hlstd{lo}
  \hlstd{diff} \hlkwb{<-} \hlstd{x} \hlopt{-} \hlstd{y}
  \hlkwd{quantile}\hlstd{(diff,} \hlkwc{probs} \hlstd{=} \hlkwd{c}\hlstd{(lo, hi))}
\hlstd{\}}
\end{alltt}
\end{kframe}
\end{knitrout}

Now we simply iterate through each of the {\tt beta.*} and report the
results. To make the coefficients that are detectably different from
one another a bit easier to spot, we add a '*' after the credible
interval.

\begin{knitrout}
\definecolor{shadecolor}{rgb}{0.969, 0.969, 0.969}\color{fgcolor}\begin{kframe}
\begin{alltt}
\hlcom{## get the indices for beta coefficients in the}
\hlcom{## objects they'll be the same for protea and}
\hlcom{## pellie}
\hlstd{idx} \hlkwb{<-} \hlkwd{grep}\hlstd{(}\hlstr{"beta"}\hlstd{,} \hlkwd{names}\hlstd{(pellie))}
\hlkwa{for} \hlstd{(i} \hlkwa{in} \hlstd{idx) \{}
  \hlkwa{for} \hlstd{(j} \hlkwa{in} \hlnum{1}\hlopt{:}\hlnum{4}\hlstd{) \{}
    \hlstd{cred} \hlkwb{<-} \hlkwd{compare}\hlstd{(protea[[i]][, j], pellie[[i]][,}
      \hlstd{j])}
    \hlkwd{cat}\hlstd{(}\hlkwd{names}\hlstd{(protea)[[i]],} \hlstr{"["}\hlstd{, j,} \hlstr{"]"}\hlstd{,} \hlstr{" ("}\hlstd{,}
      \hlstd{cred[}\hlnum{1}\hlstd{],} \hlstr{","}\hlstd{, cred[}\hlnum{2}\hlstd{],} \hlstr{")"}\hlstd{,} \hlkwc{sep} \hlstd{=} \hlstr{""}\hlstd{)}
    \hlcom{## we know that the credible interval doesn't}
    \hlcom{## overlap zero if either the lower bound is}
    \hlcom{## positive or the upper bound is negative}
    \hlkwa{if} \hlstd{((cred[}\hlnum{1}\hlstd{]} \hlopt{>} \hlnum{0}\hlstd{)} \hlopt{||} \hlstd{(cred[}\hlnum{2}\hlstd{]} \hlopt{<} \hlnum{0}\hlstd{)) \{}
      \hlkwd{cat}\hlstd{(}\hlstr{"*"}\hlstd{,} \hlkwc{sep} \hlstd{=} \hlstr{""}\hlstd{)}
    \hlstd{\}}
    \hlkwd{cat}\hlstd{(}\hlstr{"\textbackslash{}n"}\hlstd{)}
  \hlstd{\}}
\hlstd{\}}
\end{alltt}
\begin{verbatim}
## beta.cdd[1] (-0.03624,0.3088)
## beta.cdd[2] (-0.785,-0.3749)*
## beta.cdd[3] (-0.04717,0.182)
## beta.cdd[4] (-0.0724,0.2932)
## beta.elev[1] (-0.4811,0.06748)
## beta.elev[2] (-0.4061,0.2245)
## beta.elev[3] (-0.1862,0.2201)
## beta.elev[4] (-0.1839,0.3939)
## beta.inso[1] (-0.3677,-0.07219)*
## beta.inso[2] (-0.07133,0.2912)
## beta.inso[3] (-0.1027,0.1064)
## beta.inso[4] (-0.02798,0.2842)
## beta.map[1] (-0.1653,0.2921)
## beta.map[2] (-0.4748,0.07021)
## beta.map[3] (0.1035,0.424)*
## beta.map[4] (-0.271,0.1902)
## beta.mat[1] (0.2681,0.7574)*
## beta.mat[2] (-0.5638,0.05674)
## beta.mat[3] (-0.182,0.1915)
## beta.mat[4] (-0.2947,0.2045)
## beta.ratio[1] (0.01782,0.4088)*
## beta.ratio[2] (-0.4621,0.02279)
## beta.ratio[3] (-0.3998,-0.1067)*
## beta.ratio[4] (-0.2784,0.1458)
\end{verbatim}
\end{kframe}
\end{knitrout}

It's probably helpful to visualize the differences by plotting the
posteriors.

\begin{knitrout}
\definecolor{shadecolor}{rgb}{0.969, 0.969, 0.969}\color{fgcolor}\begin{kframe}
\begin{alltt}
\hlkwd{source}\hlstd{(}\hlstr{"posterior-comparisons.R"}\hlstd{)}
\end{alltt}


{\ttfamily\noindent\itshape\color{messagecolor}{\#\# Loading required package: ggplot2}}\end{kframe}
\includegraphics[width=\maxwidth]{figure/unnamed-chunk-41} 

\includegraphics[width=\maxwidth]{figure/unnamed-chunk-42} 

\includegraphics[width=\maxwidth]{figure/unnamed-chunk-43} 

\includegraphics[width=\maxwidth]{figure/unnamed-chunk-44} 

\end{knitrout}

Here's another approach: Instead of looking at the difference between
regression coefficients, we ask "Do these coefficients have the same
sign?" There are three possible answers: "(1) Yes. Both are
negative. (2) No. One is positive and the other is negative. (3)
Yes. Both are positive."

\begin{knitrout}
\definecolor{shadecolor}{rgb}{0.969, 0.969, 0.969}\color{fgcolor}\begin{kframe}
\begin{alltt}
\hlstd{compare.sign} \hlkwb{<-} \hlkwa{function}\hlstd{(}\hlkwc{x}\hlstd{,} \hlkwc{y}\hlstd{) \{}
  \hlkwd{stopifnot}\hlstd{(}\hlkwd{length}\hlstd{(x)} \hlopt{==} \hlkwd{length}\hlstd{(y))}
  \hlstd{negative} \hlkwb{<-} \hlkwd{sum}\hlstd{(}\hlkwd{as.numeric}\hlstd{((x} \hlopt{<} \hlnum{0}\hlstd{)} \hlopt{&} \hlstd{(y} \hlopt{<}
    \hlnum{0}\hlstd{)))}
  \hlstd{positive} \hlkwb{<-} \hlkwd{sum}\hlstd{(}\hlkwd{as.numeric}\hlstd{((x} \hlopt{>} \hlnum{0}\hlstd{)} \hlopt{&} \hlstd{(y} \hlopt{>}
    \hlnum{0}\hlstd{)))}
  \hlstd{conflict} \hlkwb{<-} \hlkwd{sum}\hlstd{(}\hlkwd{as.numeric}\hlstd{((x} \hlopt{<} \hlnum{0}\hlstd{)} \hlopt{&} \hlstd{(y} \hlopt{>}
    \hlnum{0}\hlstd{)))} \hlopt{+} \hlkwd{sum}\hlstd{(}\hlkwd{as.numeric}\hlstd{((x} \hlopt{>} \hlnum{0}\hlstd{)} \hlopt{&} \hlstd{(y} \hlopt{<} \hlnum{0}\hlstd{)))}
  \hlkwd{stopifnot}\hlstd{(}\hlkwd{length}\hlstd{(x)} \hlopt{==} \hlstd{(negative} \hlopt{+} \hlstd{positive} \hlopt{+}
    \hlstd{conflict))}
  \hlstd{negative} \hlkwb{<-} \hlstd{negative}\hlopt{/}\hlkwd{length}\hlstd{(x)}
  \hlstd{positive} \hlkwb{<-} \hlstd{positive}\hlopt{/}\hlkwd{length}\hlstd{(x)}
  \hlstd{conflict} \hlkwb{<-} \hlstd{conflict}\hlopt{/}\hlkwd{length}\hlstd{(x)}
  \hlkwd{list}\hlstd{(}\hlkwc{negative} \hlstd{= negative,} \hlkwc{positive} \hlstd{= positive,}
    \hlkwc{conflict} \hlstd{= conflict)}
\hlstd{\}}
\end{alltt}
\end{kframe}
\end{knitrout}

Now we iterate through each of the {\tt beta.*} as before, store the
results in a table ({\tt results}), and report the results.

\begin{knitrout}
\definecolor{shadecolor}{rgb}{0.969, 0.969, 0.969}\color{fgcolor}\begin{kframe}
\begin{alltt}
\hlcom{## get the indices for beta coefficients in the}
\hlcom{## objects they'll be the same for protea and}
\hlcom{## pellie}
\hlstd{idx} \hlkwb{<-} \hlkwd{grep}\hlstd{(}\hlstr{"beta"}\hlstd{,} \hlkwd{names}\hlstd{(pellie))}
\hlstd{coefficient} \hlkwb{<-} \hlkwd{character}\hlstd{(}\hlnum{0}\hlstd{)}
\hlstd{negative} \hlkwb{<-} \hlkwd{numeric}\hlstd{(}\hlnum{0}\hlstd{)}
\hlstd{conflict} \hlkwb{<-} \hlkwd{numeric}\hlstd{(}\hlnum{0}\hlstd{)}
\hlstd{positive} \hlkwb{<-} \hlkwd{numeric}\hlstd{(}\hlnum{0}\hlstd{)}
\hlkwa{for} \hlstd{(i} \hlkwa{in} \hlstd{idx) \{}
  \hlkwa{for} \hlstd{(j} \hlkwa{in} \hlnum{1}\hlopt{:}\hlnum{4}\hlstd{) \{}
    \hlstd{results} \hlkwb{<-} \hlkwd{compare.sign}\hlstd{(protea[[i]][,}
      \hlstd{j], pellie[[i]][, j])}
    \hlstd{coefficient} \hlkwb{<-} \hlkwd{c}\hlstd{(coefficient,} \hlkwd{paste}\hlstd{(}\hlkwd{names}\hlstd{(protea)[[i]],}
      \hlstd{j,} \hlkwc{sep} \hlstd{=} \hlstr{""}\hlstd{))}
    \hlstd{negative} \hlkwb{<-} \hlkwd{c}\hlstd{(negative, results}\hlopt{$}\hlstd{negative)}
    \hlstd{conflict} \hlkwb{<-} \hlkwd{c}\hlstd{(conflict, results}\hlopt{$}\hlstd{conflict)}
    \hlstd{positive} \hlkwb{<-} \hlkwd{c}\hlstd{(positive, results}\hlopt{$}\hlstd{positive)}
  \hlstd{\}}
\hlstd{\}}
\hlstd{results} \hlkwb{<-} \hlkwd{data.frame}\hlstd{(coefficient, negative, conflict,}
  \hlstd{positive)}
\end{alltt}
\end{kframe}
\end{knitrout}

\begin{knitrout}
\definecolor{shadecolor}{rgb}{0.969, 0.969, 0.969}\color{fgcolor}\begin{kframe}
\begin{alltt}
\hlkwd{require}\hlstd{(xtable)}
\end{alltt}


{\ttfamily\noindent\itshape\color{messagecolor}{\#\# Loading required package: xtable}}

{\ttfamily\noindent\color{warningcolor}{\#\# Warning: package 'xtable' was built under R version 3.1.1}}\begin{alltt}
\hlstd{results.table} \hlkwb{<-} \hlkwd{xtable}\hlstd{(results,} \hlkwc{digits} \hlstd{=} \hlnum{3}\hlstd{)}
\hlkwd{print}\hlstd{(results.table,} \hlkwc{file} \hlstd{=} \hlstr{"table-temp.tex"}\hlstd{,}
  \hlkwc{include.rownames} \hlstd{=} \hlnum{FALSE}\hlstd{)}
\end{alltt}
\end{kframe}
\end{knitrout}

% latex table generated in R 3.1.0 by xtable 1.7-4 package
% Thu Sep 25 19:57:56 2014
\begin{table}[ht]
\centering
\begin{tabular}{lrrr}
  \hline
coefficient & negative & conflict & positive \\ 
  \hline
beta.cdd1 & 0.043 & 0.715 & 0.242 \\ 
  beta.cdd2 & 0.000 & 0.999 & 0.001 \\ 
  beta.cdd3 & 0.787 & 0.212 & 0.001 \\ 
  beta.cdd4 & 0.040 & 0.598 & 0.362 \\ 
  beta.elev1 & 0.495 & 0.493 & 0.012 \\ 
  beta.elev2 & 0.966 & 0.034 & 0.000 \\ 
  beta.elev3 & 0.376 & 0.481 & 0.142 \\ 
  beta.elev4 & 0.164 & 0.575 & 0.261 \\ 
  beta.inso1 & 0.001 & 0.825 & 0.175 \\ 
  beta.inso2 & 0.406 & 0.572 & 0.022 \\ 
  beta.inso3 & 0.067 & 0.369 & 0.565 \\ 
  beta.inso4 & 0.424 & 0.570 & 0.005 \\ 
  beta.map1 & 1.000 & 0.000 & 0.000 \\ 
  beta.map2 & 0.769 & 0.230 & 0.001 \\ 
  beta.map3 & 0.004 & 0.973 & 0.024 \\ 
  beta.map4 & 0.000 & 0.035 & 0.965 \\ 
  beta.mat1 & 0.102 & 0.898 & 0.000 \\ 
  beta.mat2 & 1.000 & 0.000 & 0.000 \\ 
  beta.mat3 & 0.677 & 0.306 & 0.017 \\ 
  beta.mat4 & 0.000 & 0.026 & 0.974 \\ 
  beta.ratio1 & 0.000 & 0.000 & 1.000 \\ 
  beta.ratio2 & 0.014 & 0.733 & 0.253 \\ 
  beta.ratio3 & 0.131 & 0.869 & 0.000 \\ 
  beta.ratio4 & 0.809 & 0.187 & 0.004 \\ 
   \hline
\end{tabular}
\end{table}


\bibliography{posterior-comparisons}
\bibliographystyle{plain}

\end{document}
